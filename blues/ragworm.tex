% Options for packages loaded elsewhere
\PassOptionsToPackage{unicode}{hyperref}
\PassOptionsToPackage{hyphens}{url}
%
\documentclass[
]{book}
\usepackage{amsmath,amssymb}
\usepackage{lmodern}
\usepackage{iftex}
\ifPDFTeX
  \usepackage[T1]{fontenc}
  \usepackage[utf8]{inputenc}
  \usepackage{textcomp} % provide euro and other symbols
\else % if luatex or xetex
  \usepackage{unicode-math}
  \defaultfontfeatures{Scale=MatchLowercase}
  \defaultfontfeatures[\rmfamily]{Ligatures=TeX,Scale=1}
\fi
% Use upquote if available, for straight quotes in verbatim environments
\IfFileExists{upquote.sty}{\usepackage{upquote}}{}
\IfFileExists{microtype.sty}{% use microtype if available
  \usepackage[]{microtype}
  \UseMicrotypeSet[protrusion]{basicmath} % disable protrusion for tt fonts
}{}
\makeatletter
\@ifundefined{KOMAClassName}{% if non-KOMA class
  \IfFileExists{parskip.sty}{%
    \usepackage{parskip}
  }{% else
    \setlength{\parindent}{0pt}
    \setlength{\parskip}{6pt plus 2pt minus 1pt}}
}{% if KOMA class
  \KOMAoptions{parskip=half}}
\makeatother
\usepackage{xcolor}
\usepackage{longtable,booktabs,array}
\usepackage{calc} % for calculating minipage widths
% Correct order of tables after \paragraph or \subparagraph
\usepackage{etoolbox}
\makeatletter
\patchcmd\longtable{\par}{\if@noskipsec\mbox{}\fi\par}{}{}
\makeatother
% Allow footnotes in longtable head/foot
\IfFileExists{footnotehyper.sty}{\usepackage{footnotehyper}}{\usepackage{footnote}}
\makesavenoteenv{longtable}
\usepackage{graphicx}
\makeatletter
\def\maxwidth{\ifdim\Gin@nat@width>\linewidth\linewidth\else\Gin@nat@width\fi}
\def\maxheight{\ifdim\Gin@nat@height>\textheight\textheight\else\Gin@nat@height\fi}
\makeatother
% Scale images if necessary, so that they will not overflow the page
% margins by default, and it is still possible to overwrite the defaults
% using explicit options in \includegraphics[width, height, ...]{}
\setkeys{Gin}{width=\maxwidth,height=\maxheight,keepaspectratio}
% Set default figure placement to htbp
\makeatletter
\def\fps@figure{htbp}
\makeatother
\setlength{\emergencystretch}{3em} % prevent overfull lines
\providecommand{\tightlist}{%
  \setlength{\itemsep}{0pt}\setlength{\parskip}{0pt}}
\setcounter{secnumdepth}{5}
\usepackage{booktabs}
\ifLuaTeX
  \usepackage{selnolig}  % disable illegal ligatures
\fi
\usepackage[style=apa,]{biblatex}
\addbibresource{cite.bib}
\IfFileExists{bookmark.sty}{\usepackage{bookmark}}{\usepackage{hyperref}}
\IfFileExists{xurl.sty}{\usepackage{xurl}}{} % add URL line breaks if available
\urlstyle{same} % disable monospaced font for URLs
\hypersetup{
  pdftitle={Protocols for Ragworms},
  pdfauthor={Duy Nghia Pham, Inna M. Sokolova},
  hidelinks,
  pdfcreator={LaTeX via pandoc}}

\title{Protocols for Ragworms}
\author{Duy Nghia Pham, Inna M. Sokolova}
\date{2023-01-07}

\begin{document}
\maketitle

{
\setcounter{tocdepth}{1}
\tableofcontents
}
\hypertarget{homogenization}{%
\chapter{Homogenization}\label{homogenization}}

\hypertarget{materials}{%
\section{Materials}\label{materials}}

\begin{itemize}
\item
  0.1 M Tris:HCl pH 8.5 containing 153 μM MgSO4, 0.2\% w/v Triton X-100, 0.1 mM PMSF:

  \begin{itemize}
  \item
    2.5 M HCl: in 250 mL conical flask pre-filled with ≈100 mL water add 51.754 mL\footnote{\url{https://www.sigmaaldrich.com/DE/en/support/calculators-and-apps/molarity-calculator}} 37\% HCl (d = 1.19 g/mL) using glass pipet. Transfer to 250 mL volumetric flask and fill up with water. Transfer to 250 mL bottle. Store at 25°C.
  \item
    In 1 L beaker add 12.114 g\footnote{\url{https://www.sigmaaldrich.com/DE/en/support/calculators-and-apps/mass-molarity-calculator}} Tris base (M = 121.14) and 0.0377 g MgSO4*7H2O (M = 246.48) then fill with ≈800 mL water. Add 21-23 mL 2.5 M HCl to reach pH 8.5. Transfer to 1 L volumetric flask and fill up with water. Transfer to 1 L bottle then dissolve 1869 μL Triton X100 (d = 1.07 mg/μL). Store at 4°C.
  \item
    200 mM PMSF stock: in 1.5 mL tube add 0.035 g PMSF (M = 174.2) and 1 mL isopropanol, store at -20 °C.
  \end{itemize}
\item
  1.5 mL flat\footnote{compatible with FastPrep} screw cap tubes pre-filled with 500 mg zirconium beads (≈250 mg 2 mm beads and ≈250 mg 1 mm beads).
\item
  2 mL socket screw cap tubes.
\item
  Avery\footnote{\url{https://www.avery-zweckform.com/blanko-etiketten/rund-10-mm}} freeze-resistant round labels, no. 1-15 for each sample.
\end{itemize}

\hypertarget{methods}{%
\section{Methods}\label{methods}}

\begin{itemize}
\item
  Quickly cut the head (≈50 mg) and wrap it in the aluminum foil maintained in liquid nitrogen.
\item
  Record mass of the headless worm, transfer to 15 mL pre-cool glass homogenizer.
\item
  Prepare 1:25 tissue homogenate:

  \begin{itemize}
  \item
    in 50 mL tube add 2.5 μL PMSF stock to every 5 mL cold buffer (2000-fold dilution) right before use. Invert briefly.
  \item
    Add 2500 μL buffer containing PMSF to every 100 mg wet tissue.
  \end{itemize}
\item
  Record mass of the head, transfer to the 1.5 mL tube pre-filled with beads (no. 1). Cool down in liquid nitrogen then temporarily store at -20 °C.
\item
  Surround glass homogenizer by ice, rotate at 200 rpm for \textasciitilde3 min.
\item
  Aliquot 1.8 mL homogenate to 2 mL tubes (no. 2+) on ice then temporarily store at -20 °C.
\item
  Store tubes (no.1 for gene expression and no. 2+ for other biomarkers in separated cryoboxes) at -80 °C for downstream applications.
\end{itemize}

\hypertarget{cea}{%
\chapter{CEA}\label{cea}}

\hypertarget{organization}{%
\section{Organization}\label{organization}}

\begin{itemize}
\tightlist
\item
  Transfer one homogenate tube (e.g., no. 2) of all samples to new cryoboxes.
\item
  Thaw out homogenate (max. 24 samples each run, 8 samples each sub-run).
\item
  Vortex and transfer 375 μL homogenate to 2 mL tube, store at -80 °C for later analysis of lipids (read more in \ref{lipids}).
\item
  Vortex and transfer 300 μL homogenate to 1.5 mL tube, centrifuge 3,000 ×\emph{g} for 5 min at 4 °C.
\item
  Transfer supernatant to 1.5 mL tube, use for ETS on the same day (read more in \ref{ETS}) and store the leftover at -80 °C for later analysis of carbohydrates and proteins (read more in \ref{proteins} and \ref{carbs}).
\end{itemize}

\hypertarget{ETS}{%
\chapter{ETS}\label{ETS}}

Modified after \textcite{decoen1997}.

\hypertarget{materials-1}{%
\section{Materials}\label{materials-1}}

\begin{itemize}
\tightlist
\item
  Buffer solution: 0.13 M Tris:HCl pH 8.5 containing 0.3\% w/v Triton X-100

  \begin{itemize}
  \tightlist
  \item
    Add 1.574 g Tris base (M = 121.14) to conical flask then fill with \textasciitilde80 mL water. Add \textasciitilde3.1 mL 2.5 M HCl to reach pH 8.5. Add 280.4 μL Triton X100 (d = 1.07 mg/μL). Dissolve, adjust pH if needed and add water to a volume of 100 mL, aided by volumetric flask. Transfer to DURAN bottle, store cold.
  \end{itemize}
\item
  Substrate solution: 1.7 mM NADH and 0.25 mM NADPH
  Add 0.0241 g NADH (M = 709.4) and 0.0042 g NADPH (M = 833.35) to 20 mL buffer solution, store dark cold
\item
  8 mM INT: 0.2023 g INT (M = 505.7) in 50 mL water, store dark cold
\item
  For each batch of 24 samples, take out 10 mL buffer in Falcon tube, 1.3 mL Substrate in 1.5 mL tube, 5.3 mL INT in 50 mL glass bottle
\item
  0.5 M KCN (M = 65.12): 32.6 mg KCN in 1mL water, freeze at -- 20
\item
  1 mM rotenone (M= 394.42): 0.39 mg rotenone in 1ml ethanol absolute, freeze at -20
\end{itemize}

\hypertarget{methods-1}{%
\section{Methods}\label{methods-1}}

\begin{itemize}
\tightlist
\item
  Rxn in duplicate wells:

  \begin{itemize}
  \tightlist
  \item
    25 μL supernatant + 75 μL buffer + 23 μL KCN + 2 μL rotenone + 50 μL INT for self-control
  \item
    25 μL supernatant + 75 μL buffer + 25 μL substrate + 50 μL INT (Handystep) for rxn
  \end{itemize}
\item
  Measure kinetically at 490 nm in 10 min with 10s interval, at room temperature (25 °C), shake 5s before reading
\end{itemize}

\hypertarget{proteins}{%
\chapter{Proteins}\label{proteins}}

Modified after \textcite{bradford1976}.

\hypertarget{materials-2}{%
\section{Materials}\label{materials-2}}

\begin{itemize}
\tightlist
\item
  5 mg/mL BSA stock solution: e.g., 50 mg BSA in 10 mL homogenization buffer
\item
  2 mg/mL BSA stock solution: 0.5 mL 5 mg/mL BSA + 0.75 mL water
\item
  Bradford reagent
\end{itemize}

\hypertarget{methods-2}{%
\section{Methods}\label{methods-2}}

\begin{itemize}
\tightlist
\item
  dilute supernatant 8 times in 1.5 mL tube: 10 μL supernatant + 70 μL water, on ice
\item
  Prepare standard curve using 2 mg/mL BSA stock solution in 1.5 mL tubes on ice
\end{itemize}

\begin{longtable}[]{@{}lll@{}}
\toprule()
mg/mL standard & uL stock & uL water \\
\midrule()
\endhead
2 & 200 & 0 \\
1 & 100 & 100 \\
0.5 & 50 & 150 \\
0.25 & 25 & 175 \\
0.125 & 12.5 & 187.5 \\
0.0625 & 6.25 & 193.75 \\
0 & 0 & 200 \\
\bottomrule()
\end{longtable}

\begin{itemize}
\tightlist
\item
  Add 200 μL Bradford reagent using handystep to 10 μL diluted supernatant or standards in wells
\item
  Shaking 300s (5 min) and measure in triplicate at 595 nm
\end{itemize}

\hypertarget{carbs}{%
\chapter{Carbohydrates}\label{carbs}}

Modified after \textcite{masuko2005}.

\hypertarget{materials-3}{%
\section{Materials}\label{materials-3}}

\begin{itemize}
\tightlist
\item
  Concentrated H2SO4, store dark room
\item
  5\% phenol: 2 mL 90\% phenol + 34 mL water in Falcon tube, vortex then transfer to 50 mL glass bottle, store dark
\item
  100 mM glucose stock solution: e.g., 180.16 mg in 10 mL water, store cold
\item
  2 mM glucose stock solution: 200 μL stock solution + 9800 μL water, store cold
\end{itemize}

\hypertarget{methods-3}{%
\section{Methods}\label{methods-3}}

\begin{itemize}
\tightlist
\item
  dilute supernatant 8 times in 2 mL tube: 25 μL supernatant + 175 μL water = 200 μL
\item
  Prepare standard curve using 2 mM stock solution in 2 mL tubes: 2, 1, 0.5, 0.25, 0.125, 0.0625, 0 mM (200 μL in final)
\end{itemize}

\begin{longtable}[]{@{}lll@{}}
\toprule()
mM standard & uL stock & uL water \\
\midrule()
\endhead
2 & 200 & 0 \\
1 & 100 & 100 \\
0.5 & 50 & 150 \\
0.25 & 25 & 175 \\
0.125 & 12.5 & 187.5 \\
0.0625 & 6.25 & 193.75 \\
0 & 0 & 200 \\
\bottomrule()
\end{longtable}

\begin{itemize}
\tightlist
\item
  Medium vortex samples and standards
\item
  Add 500 μL concentrated H2SO4 and 100 μL 5\% phenol both using Handystep
\item
  VORTEX FOR A WHILE and heat at 90 °C for 5 min, cool down for 5 min then vortex.
\item
  measure 200 μL in triplicate at 492 nm. Shake 5s before reading.
\end{itemize}

\hypertarget{lipids}{%
\chapter{Lipids}\label{lipids}}

Modified after \textcite{inouye2006}.

\hypertarget{materials-4}{%
\section{Materials}\label{materials-4}}

\begin{itemize}
\tightlist
\item
  Chloroform: methanol 1:1 v/v: mix 150 mL chloroform and 150 mL methanol in a 500 mL bottle, store dark room, pour to 100 mL bottle to use
\item
  Concentrated H2SO4, store dark room
\item
  5 mg oil/mL acetone stock solution: 5.7 μL oil (d = 0.87 mg/μL) in 1 mL acetone\footnote{Pre-wetting 1 mL tip can prevent acetone dropping}
\item
  Vanillin reagent: dissolve 750 mg vanillin in 125 mL hot water then add 500 mL 85\% H3PO4 in a 1 L bottle, store dark room, pour to 250 mL bottle to use
\end{itemize}

\hypertarget{methods-4}{%
\section{Methods}\label{methods-4}}

\begin{itemize}
\tightlist
\item
  Prepare standard curve using stock solution in 1.5 mL tubes: 5, 2.5, 1.25, 0.625, 0.3125, 0 mg/mL
\end{itemize}

\begin{longtable}[]{@{}lll@{}}
\toprule()
mg/mL standard & uL stock & uL acetone \\
\midrule()
\endhead
5 & 200 & 0 \\
2.5 & 100 & 100 \\
1.25 & 50 & 150 \\
0.625 & 25 & 175 \\
0.3125 & 12.5 & 187.5 \\
0 & 0 & 200 \\
\bottomrule()
\end{longtable}

\begin{itemize}
\tightlist
\item
  Add 375 μL homogenate to 2 mL tube\footnote{already done when preparing for ETS assay}
\item
  thawing homogenate, medium vortex, add 1.5 mL Chloroform: methanol, vortex and incubate 10 min
\item
  Centrifuge 3000 g for 5 min at 4 °C, wait some time for methanol going up
\item
  transfer 200 μL of low chloroform phase \footnote{wait for methanol getting out of the tip} to 1.5 mL tube
\item
  spin down standards and samples
\item
  evaporate solvent in standards and samples by heating at 90 °C for 10 min (open cap)
\item
  Add 100 μL concentrated H2SO4 using Handystep, vortex twice and heat at 100 °C for 10 min (close cap)
\item
  After cooling, transfer 50 μL to 1.5 mL microtubes tube already filled with 1 mL vanillin reagent, vortex
\item
  After 5 min, measure 200 μL in triplicate at 490 nm
\end{itemize}

\hypertarget{mgo}{%
\chapter{MGO}\label{mgo}}

Modified after \textcite{mitchel1977}.

\hypertarget{materials-5}{%
\section{Materials}\label{materials-5}}

\begin{itemize}
\tightlist
\item
  ≈0.1 M borax: in 50 mL tube add 1.5255 g borax (M\footnote{borax decahydrate} = 381.37) and 40 mL water. Make another one. Rotate 20 min. Transfer to 100 mL bottle. Store in dark at 25°C\footnote{avoid lower temperature to prevent precipitation}.
\item
  0.2 M Girard's reagent T: in 50 mL tube add 1.3411 g Girard T (M = 167.64) and 40 mL water. Rotate 5 min. Transfer to 50 mL bottle. Store in dark at 4°C.
\item
  1232 uM methylglyoxal stock solution: in 50 mL tube add 10 uL ≈6.16 M methylglyoxal (M = 72.06, 40\% w/w in water, d = 1.11) and 50 mL homogenization buffer\footnote{0.1 M Tris:HCl pH 8.5 containing 153 uM MgSO4, 0.2\% w/v Triton X-100, no PMSF and EDTA} (5000-fold dilution). Rotate 5 min. Store at 4°C.
\end{itemize}

\hypertarget{methods-5}{%
\section{Methods}\label{methods-5}}

\begin{itemize}
\tightlist
\item
  Thaw out homogenate (max. 24 samples each run). \textbf{Vortex} and transfer 250 uL to 1.5 mL tube. Centrifuge at 10,000 ×\emph{g} for 15 min at 25°C. Transfer 200 uL supernatant to 1.5 mL tube.
\item
  Prepare 200 uL methylglyoxal standards in 1.5 mL tube by stock dilution.
\end{itemize}

\begin{longtable}[]{@{}lll@{}}
\toprule()
uM standard & uL stock & uL homogenization buffer \\
\midrule()
\endhead
1232 & 200 & 0 \\
616 & 100 & 100 \\
308 & 50 & 150 \\
154 & 25 & 175 \\
77 & 12.5 & 187.5 \\
38.5 & 6.25 & 193.75 \\
0 & 0 & 200 \\
\bottomrule()
\end{longtable}

\begin{itemize}
\tightlist
\item
  Add 250 uL borax and 50 uL Girard T to samples and standards using repeating pipette.
\item
  \textbf{Vortex} and heat at 60°C\footnote{trials at 30°C and 90°C showed lower absorbance} for 10 min. Cool to 25°C for 5 min.
\item
  Centrifuge at 10,000 ×\emph{g} for 5 min at 25°C\footnote{may need another centrifuge for standards; small pellet could appear in samples but not in standards}.
\item
  Measure\footnote{pre-wetting the tip helps minimize Triton bubbles and increase homogeneity in samples (since supernatant is taken directly for measurement without transferring to new tube and vortexing)} absorbance of 200 μL in \textbf{duplicate} at 325 nm. Shake 5s before reading.
\end{itemize}

\hypertarget{MDA}{%
\chapter{MDA}\label{MDA}}

Modified after \textcite{buege1978}.

\hypertarget{materials-6}{%
\section{Materials}\label{materials-6}}

\begin{itemize}
\tightlist
\item
  15\% w/v TCA - 0.375\% w/v TBA - 0.25 M HCl reagent: Add 0.1875 g TBA in 50 mL tube. Add 34 mL 20\% w/w TCA (density 1.1 g/mL, equivalent to 22\% w/v). Add 5 mL 2.5 M HCl. Add 10 mL water to a volume of nearly 50 mL. Wrap by aluminum, rotate for 1h, store dark room. Rotate for 5 min before use each day.
\end{itemize}

\hypertarget{methods-6}{%
\section{Methods}\label{methods-6}}

\begin{itemize}
\tightlist
\item
  Thaw out homogenate (max. 22 samples each run). \textbf{Vortex} and transfer 200 uL to 1.5 mL tube \textbf{with snap cap with hole made by needle}. Use 200 uL homogenization buffer for control in \textbf{duplicate}.
\item
  Add 500 uL reagent to samples and controls (24 tubes in total), \textbf{NOT vortex}.
\item
  Heat at 90 °C for 20 min. Cool for 15 min.
\item
  Centrifuge at 10,000 ×\emph{g} for 3 min at 25 °C. Rotate tubes 180° then centrifuge one more time with same settings.
\item
  Allow air to escape (especially in control). Measure absorbance of 200 μL in \textbf{triplicate} at 530 nm. Shake 5s before reading.
\end{itemize}

\hypertarget{enzyme-assays}{%
\chapter{Enzyme assays}\label{enzyme-assays}}

\hypertarget{organization-1}{%
\section{Organization}\label{organization-1}}

\begin{itemize}
\tightlist
\item
  Thaw out homogenate (max. 22 samples each run, 11 samples each sub-run).
\item
  Vortex and transfer 300 μL homogenate to 1.5 mL tube, centrifuge 6,000 ×\emph{g} for 10 min at 4 °C.
\item
  Transfer supernatant to 1.5 mL tube, use for GR on the same day (read more in \ref{GR}) and store the leftover at -80 °C for later measurement of GST (read more in \ref{GST}).
\end{itemize}

\hypertarget{GR}{%
\chapter{GR}\label{GR}}

Modified after \textcite{mannervik1999}.

\hypertarget{materials-7}{%
\section{Materials}\label{materials-7}}

\begin{itemize}
\tightlist
\item
  1 mM NADPH: in 5 mL tube add 4.2 mg NADPH (M = 833.4) and 5 mL homogenization buffer. Store at 4°C.
\item
  10 mM GSSG: in 5 mL tube add 30.6 mg GSSG (M = 612.6) and 5 mL water. Store at 4°C.
\item
  1\% BSA: in 5 mL tube add 50 mg BSA and 5 mL homogenization buffer. Store at 4°C.
\end{itemize}

\hypertarget{methods-7}{%
\section{Methods}\label{methods-7}}

\begin{itemize}
\tightlist
\item
  Master mix for 24 rxn: in 5 mL tube add 3120 uL homogenization buffer, 520 uL NADPH, 520 uL GSSG, and 520 uL BSA (120:20:20:20). \textbf{Vortex}.
\item
  Rxn in duplicate wells:

  \begin{itemize}
  \tightlist
  \item
    180 uL master mix + 20 uL homogenization buffer for control
  \item
    180 uL master mix + 20 uL supernatant for sample
  \end{itemize}
\item
  Measure kinetically at 340 nm in 15 min with 30s interval, at room temperature (25 °C), shake 5s before reading. Use 21/31 points for slope.
\end{itemize}

\hypertarget{GST}{%
\chapter{GST}\label{GST}}

Modified after \textcite{habig1974}.

\hypertarget{materials-8}{%
\section{Materials}\label{materials-8}}

\begin{itemize}
\tightlist
\item
  5 mM GSH: in 5 mL tube add 7.7 mg GSH (M = 307.33) and 5 mL homogenization buffer. Use fresh only.
\item
  5 mM CDNB:

  \begin{itemize}
  \tightlist
  \item
    Prepare 100 mM CNDB stock: e.g., in 15 mL tube add 286.9 mg CDNB (M = 202.55) and 14.16 mL absolute ethanol. Store at -20°C.
  \item
    Dilute 20X: in 2mL tube add 50 uL stock and 950 uL absolute ethanol.
  \end{itemize}
\end{itemize}

\hypertarget{methods-8}{%
\section{Methods}\label{methods-8}}

\begin{itemize}
\tightlist
\item
  Master mix for 24 rxn: in 5 mL tube add 4160 uL homogenization buffer, 260 uL GSH, and 260 uL CDNB (160:10:10). \textbf{Vortex}.
\item
  Rxn in duplicate wells:

  \begin{itemize}
  \tightlist
  \item
    180 uL master mix + 20 uL homogenization buffer for control
  \item
    180 uL master mix + 20 uL supernatant for sample
  \end{itemize}
\item
  Measure kinetically at 340 nm in 15 min with 30s interval, at room temperature (25 °C), shake 5s before reading. Use 21/31 points for slope.
\end{itemize}

\hypertarget{tac}{%
\chapter{TAC}\label{tac}}

Modified after \textcite{re1999}.

\hypertarget{materials-9}{%
\section{Materials}\label{materials-9}}

\begin{itemize}
\tightlist
\item
  ≈63.63 mM potassium persulfate: in 5 mL tube add 86 mg K\textsubscript{2}S\textsubscript{2}O\textsubscript{8} (M = 270.32) and 5 mL water. Vortex.
\item
  ≈7.29 mM ABTS: in 15 mL tube add 40 mg ABTS salt (M = 548.68) and 10 mL water. Vortex.
\item
  ABTS radical cation stock solution: add 0.4 mL K\textsubscript{2}S\textsubscript{2}O\textsubscript{8} to 10 mL ABTS (final conc. 2.45 mM K\textsubscript{2}S\textsubscript{2}O\textsubscript{8} + 7 mM ABTS). Vortex. Store in dark at 25°C for ≈24h before use.
\item
  10 mM Trolox: in 5 mL tube add 12.5 mg Trolox (M = 250.29) and 5 mL homogenization buffer. Rotate 5 min. Use fresh.
\end{itemize}

\hypertarget{methods-9}{%
\section{Methods}\label{methods-9}}

\begin{itemize}
\tightlist
\item
  Thaw out supernatant (max. 40 samples each run). \textbf{Vortex} before use.
\item
  Prepare 200 uL Trolox standards in 1.5 mL tube by stock dilution. \textbf{Vortex} and keep on ice.
\end{itemize}

\begin{longtable}[]{@{}lll@{}}
\toprule()
uM standard & uL stock & uL homogenization buffer \\
\midrule()
\endhead
0 & 0 & 200 \\
500 & 10 & 190 \\
1000 & 20 & 180 \\
1500 & 30 & 170 \\
2000 & 40 & 160 \\
2500 & 50 & 150 \\
3000 & 60 & 140 \\
\bottomrule()
\end{longtable}

\begin{itemize}
\tightlist
\item
  Dilute\footnote{check Abs beforehand with 3000 and 0 uM standards to fit the reader capacity. Adjust by adding stock solution or water} ABTS\textsuperscript{•+} ≈10X: in 50 mL tube add, e.g., 2.5 mL ABTS\textsuperscript{•+} stock solution and 22.5 mL water. Vortex.
\item
  Rxn in duplicate wells: 10 uL standards or samples + 200 uL diluted ABTS\textsuperscript{•+} using repeating pipette.
\item
  Measure endpoint at 734 nm at room temperature (25 °C), shake 300s (5 min) before reading.
\end{itemize}

\hypertarget{gene-expression}{%
\chapter{Gene expression}\label{gene-expression}}

updating

\printbibliography

\end{document}
